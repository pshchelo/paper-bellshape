\documentclass[10pt,a4paper,draft]{article}
\usepackage[utf8]{inputenc}
\usepackage{amsmath}
\usepackage{amsfonts}
\usepackage{amssymb}
\usepackage{makeidx}
\usepackage{pshchelo}
\bibliographystyle{unsrt}
\usepackage[pdftex]{graphicx}

\begin{document}
\author{Pavlo Shchelokovskyy}
\title{Instability of bell shape in vesicles sedimenting under gravity}
\date{\today}
\maketitle

\abstract{The bell like shape for vesicles sedimenting under the gravity predicted previousely was observed.
The shape is in the agreement with the model given experimental parameters.
However it was found to be unstable contrary to theoretical predictions.
This instability may be attributed to thermal undulations of lipid mambrane.}

\section{Introduction}\label{intro}
Introduction

\section{Materials and methods}\label{methods}
\subsection{Vesicle preparation}
Vesicles were prepared from DOPC by electroformation method \cite{Angelova1986} in 300~mM sucrose.
Afterwards they were diluted with solution of 335~mM glucose + 0.01~mM NaCl to provide for density, osmolarity, conductivity and optical density contrast in the proportion of Vesicle solution to Glucose-NaCl solution as 1:10.
This mix was then left for 15 min to sediment in the tube, separating bigger vesicles from the rest.
The final sample was collected form the bottom of the tube, and filled in the experimental chamber.
After that the unsealed chamber was left to evaporate in the room conditions for another 15 minutes in order to slowly change the osmolarity contrast and thus gently deflate vesicles.
Then the chamber was sealed.

\subsection{Electrodeformation of sedimenting vesicles}
In order to obtain initial condition for sedimenting vesicles as oblate ellipsoid we used method of electrodefirmation in the AC field utilizing effect described in \cite{Aranda2008}.
We used a custom made chamber.
By using the AC field of frequency 100~kHz and the geometry of our chamber we were able to obtain the vesicles with oblate shape, with axis parallel to the gravity direction.

\section{Results}\label{results}
The vesicles starting with oblate shape do indeed initially exhibit the bell-like shape when sedimenting as predicted in \cite{Boedec2012}.
But it turns out that this morphology is not stable. After some time vesicles morph to pear-like shape (sometimes with pearling).

\section{Discussion}\label{discussion}
The course of this change strongly suggests that thermal undulations of the membrane, which were not considered in the modelling of \cite{Boedec2012}, indeed play a crucial role, destabilizing the bell-like shape and making it a meta-stable morphology.

\bibliography{pshchelo}
\end{document}